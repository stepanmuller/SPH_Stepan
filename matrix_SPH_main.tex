\documentclass{article}
\usepackage{graphicx} % Required for inserting images
\usepackage[a4paper, margin=1in]{geometry}
\usepackage{amsmath}

\title{Matrix Based SPH Formulation}
\author{Štěpán Müller}
\date{May 2024}

\begin{document}

\maketitle

\section{Introduction}
In this article, a matrix-based formulation of a numerical SPH scheme is presented. 
For demonstration, the following weakly compressible SPH scheme in 2 dimensions will be used:
$$\frac{D\rho_i}{Dt}=-\rho_i\sum_j\left(\textbf{u}_j-\textbf{u}_i\right)\cdot\nabla W_{ij}V_j+c_0\sum_j\left(\rho_j-\rho_i\right)\frac{\left(\textbf{r}_j-\textbf{r}_i\right)\cdot \nabla W_{ij}}{\mid\mid\left(\textbf{r}_j-\textbf{r}_i\right)\mid\mid^2}V_j$$

$$\frac{D\textbf{u}_i}{Dt}=-\frac{1}{\rho_i}\sum_j\left(p_j+p_i\right)\cdot\nabla W_{ij}V_j+\textbf{f}_i+\nu\frac{\rho_0}{\rho_i}\sum_j\pi_{ij}\nabla W_{ij}V_j$$

$$\frac{D\textbf{r}_i}{Dt}=\textbf{u}_i$$

$$p_i=c_0^2\left(\rho_i-\rho_0\right)$$

where $W_{ij}$ is the Gaussian Kernel, resulting in

$$\nabla W_{ij}= 
\begin{bmatrix}
x_j - x_i \\
y_j - y_i \\
\end{bmatrix}
\cdot\frac{2}{\pi h^4}\cdot exp\left(-\frac{\left(x_j-x_i\right)^2+\left(y_j-y_i\right)^2}{h^2}\right)$$

and $\pi_{ij}$ equals to
$$\pi_{ij}=8\cdot\frac{\left(\textbf{u}_j-\textbf{u}_i\right)\cdot\left(\textbf{r}_j-\textbf{r}_i\right)}{\mid\mid\textbf{r}_j-\textbf{r}_i\mid\mid^2}$$
A common algorithmization approach is to loop for $i$ over all particles and calculate their interactions with appropriate $j$ particles. The aim of the matrix based method presented in this article is to greatly reduce the amount of operations performed in a loop and replace them with matrix and vector operations.

\section{Constructing the matrix based formulation}
\subsection{Interaction matrix I}
The matrix method starts by identifying possible interacting particle pairs (for example using the bin sort method) and storing this information in the interation matrix $I$. $I$ is a sparse square symetrical binary matrix of size $N\times N$, where $N$ is the number of particles.
$$I = 
\begin{bmatrix}
0 & 1 & 0\\
1 & 0 & 1\\
0 & 1 & 0\\
\end{bmatrix}
$$
Each row $i$ represents the interactions of particle $i$ with particles $j$, where $j$ is the column. Here the first row would mean that particle $1$ does not interact with itself, it interacts with particle $2$ and does not interact with particle $3$. 

$I$ can be stored for example in the COO sparse format. Finding the matrix $I$ is the only loop necessary to algorithmize the process.
\subsection{Vectors of variables}
The particle variables will be referred to as 1D vectors of length $N$, for example vector
$$
\vec{x}=
\begin{bmatrix}
x_1 & x_2 & x_3\\
\end{bmatrix}
$$
stores the $x$ coordinate of $N=3$ particles.
\subsection{Matrices of differences} \label{matrices of differences}
Matrices of differences will replace all members that involve $j-i$ subtraction, such as $x_j-x_i$.
At first a matrix $Mx$ is defined, which results from a column-wise multiplication of $\Vec{x}$ and $I$, such as
$$
Mx=
\begin{bmatrix}
x_1 & x_2 & x_3\\
\end{bmatrix}
\begin{bmatrix}
0 & 1 & 0\\
1 & 0 & 1\\
0 & 1 & 0\\
\end{bmatrix}
=
\begin{bmatrix}
0 & x_2 & 0\\
x_1 & 0 & x_3\\
0 & x_2 & 0\\
\end{bmatrix}
$$
The matrix of differences $Dx$ is defined:
$$
Dx=Mx-Mx^T=
\begin{bmatrix}
0 & x_2 & 0\\
x_1 & 0 & x_3\\
0 & x_2 & 0\\
\end{bmatrix}
-
\begin{bmatrix}
0 & x_1 & 0\\
x_2 & 0 & x_2\\
0 & x_3 & 0\\
\end{bmatrix}
=
\begin{bmatrix}
0 & x_2-x_1 & 0\\
x_1-x_2 & 0 & x_3-x_2\\
0 & x_2-x_3 & 0\\
\end{bmatrix}
$$
Using the same approach, difference matrices $Dy, Du, Dv, D\rho$ can be calculated, with the exception of difference matrix $Dp$, which uses addition instead of subtraction due to the chosen SPH scheme.

\subsection{Matrix $\nabla W_x$ and $\nabla W_y$}
Hereafter, a simplified notation will be used:
$$ x_j-x_i=dx_{ij} $$
Steps to construct the $\nabla W$ matrices follow:
$$
Mr^2 = Dx^2 + Dy^2 = 
\begin{bmatrix}
0 & x_{12}^2 & 0\\
x_{21}^2 & 0 & x_{23}^2\\
0 & x_{32}^2 & 0\\
\end{bmatrix}
+
\begin{bmatrix}
0 & y_{12}^2 & 0\\
y_{21}^2 & 0 & y_{23}^2\\
0 & y_{32}^2 & 0\\
\end{bmatrix}
=
\begin{bmatrix}
0 & r_{12}^2 & 0\\
r_{21}^2 & 0 & r_{23}^2\\
0 & r_{32}^2 & 0\\
\end{bmatrix}
$$
Using the exponential function, only non-zero elements are taken into account:
$$
\nabla W_{size} = exp_{arg\neq0}\left(\frac{-1}{h^2}Mr^2\right)\cdot\frac{2}{\pi h^4}
$$
Eventually, matrices $\nabla W_x$ and $\nabla W_y$ can be acquired using element-wise multiplication:
$$
\nabla W_x = Dx\nabla W_{size} = 
\begin{bmatrix}
0 & \frac{2dx_{12}}{\pi h^4}\cdot exp\left(-\frac{dx_{12}^2+dy_{12}^2}{h^2}\right) & 0\\
\cdots & 0 & \cdots\\
0 & \cdots & 0\\
\end{bmatrix}
$$

$$
\nabla W_y = Dy\nabla W_{size} = 
\begin{bmatrix}
0 & \frac{2dy_{12}}{\pi h^4}\cdot exp\left(-\frac{dx_{12}^2+dy_{12}^2}{h^2}\right) & 0\\
\cdots & 0 & \cdots\\
0 & \cdots & 0\\
\end{bmatrix}
$$
\subsection{Matrix $\pi$}
Similarly to construction of $\nabla W$, a matrix $\pi$ can be constructed using element-wise multiplications:
$$\pi = 8\cdot\left(DuDx + DvDy\right)\cdot\frac{1}{Mr^2}$$
\section{Final matrix SPH scheme}
For the chosen SPH scheme, the matrix formulation gives:
$$
\overrightarrow{\frac{D\rho}{Dt}}=-\Vec{\rho}\left[\left(Du\nabla W_x + Dv \nabla W_y\right)\cdot\Vec{V}\right]
+
c_0\left[\frac{D\rho\left(Dx\nabla W_x + Dy\nabla W_y \right)}{\sqrt{Mr^2}}\right]\cdot\Vec{V}
$$
$$
\overrightarrow{\frac{Du}{Dt}}=\frac{-1}{\Vec{\rho}}\left[\left(Dp\nabla W_x\right)\cdot\Vec{V}\right]+\frac{\nu\rho_0}{\Vec{\rho}}\left[\left(\pi\nabla W_x\right)\cdot\Vec{V}\right]+\Vec{g_x}
$$
$$
\overrightarrow{\frac{Dv}{Dt}}=\frac{-1}{\Vec{\rho}}\left[\left(Dp\nabla W_y\right)\cdot\Vec{V}\right]+\frac{\nu\rho_0}{\Vec{\rho}}\left[\left(\pi\nabla W_y\right)\cdot\Vec{V}\right]+\Vec{g_y}
$$
$$\overrightarrow{\frac{Dx}{Dt}}=\Vec{u}$$
$$\overrightarrow{\frac{Dy}{Dt}}=\Vec{v}$$
$$
\vec{p}=c_0^2\left(\Vec{\rho}-\Vec{\rho_0}\right)
$$
where the $\cdot$ symbol indicates regular matrix by vector multiplication, whereas no dot suggests element-wise multiplication.
\section{Optimization}
\subsection{Data vector format}
Because matrices 
$$Dx, Dy, Du, Dv, Dp, Mr^2, \nabla W_{size}, \nabla W_x, \nabla W_y, \pi$$
all have the same structure, sharing the row and column indices of $I$, most operations can be only performed on the data vector using the COO matrix format. Using the example notation
$$\vec{Dx} = Dx.data$$
the scheme can be rewritten to a more efficient vector format. The only need to reuse the full matrix format comes from the $\cdot$ multiplication by vector of volumes $\Vec{V}$:

$$
\overrightarrow{\frac{D\rho}{Dt}}
=
-\Vec{\rho}\left[to\_coo\_matrix\left(\vec{Du}\vec{\nabla W_x} + \vec{Dv} \vec{\nabla W_y}\right)\cdot\Vec{V}\right]
+
to\_coo\_matrix\left[\frac{c_0 \Vec{D\rho}\left(\Vec{Dx}\Vec{\nabla W_x} + \Vec{Dy}\Vec{\nabla W_y} \right)}{\sqrt{\Vec{Mr^2}}}\right]\cdot\Vec{V}
$$
$$
\overrightarrow{\frac{Du}{Dt}}
=
\frac{1}{\Vec{\rho}}\left[to\_coo\_matrix\left(\vec{-Dp}\vec{\nabla W_x}+\nu\rho_0\vec{\pi}\vec{\nabla W_x}\right)\cdot\Vec{V}\right]+\Vec{g_x}
$$
$$
\overrightarrow{\frac{Dv}{Dt}}
=
\frac{1}{\Vec{\rho}}\left[to\_coo\_matrix\left(\vec{-Dp}\vec{\nabla W_y}+\nu\rho_0\vec{\pi}\vec{\nabla W_y}\right)\cdot\Vec{V}\right]+\Vec{g_y}
$$
\subsection{Simplified way to calculate data vectors}
Furthermore, there is a simplier way to calculate the difference data vectors $\Vec{Dx}, \Vec{Dy}, \Vec{Du}, \Vec{Dv}, \Vec{D\rho}, \Vec{Dp}$ by only using row and column vector of the sparse COO matrix $I$, instead of using full sparse multiplication:
$$\Vec{Dx} = \Vec{x}[columns]-\Vec{y}[rows] $$
Where $\Vec{x}[columns]$ means selecting $x_i$ so that $i$ goes through all column indices of $I$. Using $I$ from previous chapters as example:
$$I = 
\begin{bmatrix}
0 & 1 & 0\\
1 & 0 & 1\\
0 & 1 & 0\\
\end{bmatrix}
\to I.rows = 
\begin{bmatrix}
1 & 2 & 2 & 3\\
\end{bmatrix}
; I.columns = 
\begin{bmatrix}
2 & 1 & 3 & 2\\
\end{bmatrix}
; I.data = 
\begin{bmatrix}
1 & 1 & 1 & 1\\
\end{bmatrix}
$$
$$
\vec{x}=
\begin{bmatrix}
x_1 & x_2 & x_3\\
\end{bmatrix}
; 
\vec{x}[columns]=
\begin{bmatrix}
x_2 & x_1 & x_3 & x_2\\
\end{bmatrix}
; 
\vec{x}[rows]=
\begin{bmatrix}
x_1 & x_2 & x_2 & x_3\\
\end{bmatrix}
$$
$$
\Vec{Dx} = 
\begin{bmatrix}
x_2-x_1 & x_1-x_2 & x_3-x_2 & x_2-x_3\\
\end{bmatrix}
$$
As shown, this method provides the same result for $\Vec{Dx}$ as the method from \ref{matrices of differences}.
\section{Note about performance}
Compared to the standard looping method, which only requires to store vectors as long as $N$, the matrix based method requires to write vectors of length equal to number of particle pairs. Length of those vectors might range from $10N$ to $100N$.

Despite the memory downside, removing most of the iterative loop using the matrix formulation could provide advantages in cases where memory is not the limit, and computational speed has the highest priority. Implementing the matrix method using efficient multiprocessing for matrix/vector operations needs to be investigated in order to evaluate practical benefits.
\section{Conclusion}
A matrix based formulation of a chosen SPH scheme was discussed. The principle of the matrix method was explained. A final matrix SPH scheme was derived. Two ways for optimization of the algorithm were shown. Expected performance of the method was briefly discussed. More research and practical implementation is expected to be necessary for proper evaluation.
\end{document}
